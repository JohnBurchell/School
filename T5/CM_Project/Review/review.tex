\documentclass[conference]{IEEEtran}
\usepackage{cite}

\begin{document}
%
% paper title
% can use linebreaks \\ within to get better formatting as desired
\title{Review of Group 7's Report}

\author{\IEEEauthorblockN{
  John Burchell\IEEEauthorrefmark{1}}

  \IEEEauthorblockA{\IEEEauthorrefmark{1}john.a.burchell@gmail.com}
}

% make the title area
\maketitle

\section{Description}
The paper discusses the importance of requirements engineering from a change management perspective. The authors discuss what requirements are, how they change, what theories are present from the DIT035 and how they relate to change of requirements. The authors attempt to investigate the aforementioned theories and how they are applied to handle the change of requirements at a company named ``FindWise''. They introduce the company and their way of handling changing. The authors then analyse the data from interviews conducted at the company and contrast them with the theories discussed previously in relation to change management and change of requirements. They then discuss the findings in relation to communication and requirements management culminating in a conclusion where give suggestions in the use of the Agile development process and how to manage software requirements by giving an example of how their industry partner manages it themselves. Finally they highlight the benefits of change management. 
\section{Positive Aspects}
The presentation and description of the company ``FindWise'' in section IV was good and interesting to read. The authors made the paper more interesting with the case studies and anecdotal testimonies given by their contact at the company and they reflected on these throughout the paper. The authors included many theories from DIT035 and explained some of them in detail.

The questions that the authors wish to answer are introduced early in the introduction and a way to solve them accompanies the questions. A brief overview of the sections is included. 
\section{Suggested Improvements}

There are many improvements that could be made to this paper. For ease of reading I will separate the next section into subsections pertaining to specific areas of the paper.

\subsection*{Content}

The title does not really grab the attention of the reader, perhaps something akin to ``Managing Changing Requirements: A Case Study''. A more engaging title would make the paper more attractive to read.

The abstract for the paper starts off with a very broad and in my opinion well known statement. Anyone even remotely tied to the IT industry knows that software development has grown rapidly and that it has some major problems. I would argue that this topic isn't very interesting The abstract doesn't line up very well with the conclusion given at the end of the document. 

The rest of the introduction feels a little bit messy and there's no clear flow or aim that the text seems to be heading towards. The theories (having read about them during the course myself) seem a little forced to fit the view that the authors seem to have. I would suggest that they could be changed to flow together better.

Section V seems to again be a bit strange to read, the text is difficult to follow at times and the flow seems to diverge frequently. I don't think that the theories need to be described in detail here, provided they are referenced correctly it's enough to mention them and how they are relative.

The discussion seems to lack any relation to the data gathered, claims are made but are unsubstantiated most of the time. In saying this it reads easier than the previous sections but it could do with some more work. 

The conclusion fails to answer the questions presented in the introduction. As mentioned before it doesn't follow the abstract either. The headers are a little distracting from the text and the flow is hard to follow. The one claim made in the conclusion isn't done so until the end of the whole section and there is not much evidence presented to back the claim up. I would suggest using some of the statistics given in the appendix somewhere in the paper.

\subsection*{Layout}
The layout of the some of text was peculiar. Pay attention to this as I found it slightly distracting at times. Spacing between paragraphs was inconsistent, in places you have a single space and others two (If this was the auto-formatting I apologise, but you should attempt to fix it). 
\subsection*{Grammar and readability}
Grammar was a rather large Issue, I understand that you might be native English speakers but some more care could have been taken to ensure that the text was grammatically correct. It really did affect the ability to read the paper as often I was forced to backtrack and re-read sections due to the poor grammar. Perhaps ask a native speaker or others to read through your work next time to check for grammatical errors. I would also suggest being careful with quotation marks, I found one instance where they were not closed off and it gave the impression of a very long quote.

% that's all folks
\end{document}


