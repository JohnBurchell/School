\documentclass[conference]{IEEEtran}
\usepackage{cite}

\begin{document}
%
% paper title
% can use linebreaks \\ within to get better formatting as desired
\title{Review of Group 7's Report}

\author{\IEEEauthorblockN{
  John Burchell\IEEEauthorrefmark{1}}

  \IEEEauthorblockA{\IEEEauthorrefmark{1}john.a.burchell@gmail.com}
}

% make the title area
\maketitle

\section{Description}
The paper discusses the importance of requirements engineering from a change management perspective. The authors discuss what requirements are, how they change, what theories are present from the DIT035 and how they relate to change of requirements. The authors attempt to investigate the aforementioned theories and how they are applied to handle the change of requirements at a company named ``FindWise''. They introduce the company and their way of handling changing. The authors then analyse the data from interviews conducted at the company and contrast them with the theories discussed previously in relation to change management and change of requirements. They then discuss the findings in relation to communication and requirements management culminating in a conclusion where give suggestions in the use of the Agile development process and how to manage software requirements by giving an example of how their industry partner manages it themselves. Finally they highlight the benefits of change management. 
\section{Positive Aspects}
The presentation and description of how ``FindWise'' works was done well. The authors made the paper more interesting with the case studies and anecdotal testimonies given by their contact at the company and they reflected on these throughout the paper. The authors included many theories from DIT035 and explained some of them in detail.

The questions that the authors wish to answer are introduced early in the introduction and a way to solve them accompanies the questions. A brief overview of the sections is included. 
\section{Suggested Improvements}

There are many improvements that could be made to this paper. For ease of reading I will separate the next section into subsections pertaining to specific areas of the paper.

\subsection*{Content}
The abstract for the paper starts off with a very broad and in my opinion well known statement. Anyone even remotely tied to the IT industry knows that software development has grown rapidly and that it has some major problems. The abstract doesn't line up very well with the conclusion given at the end of the document. 

The rest of the introduction feels a little bit messy and there's no clear flow or aim that the text seems to be heading towards. The theories (having read about them during the course myself) seem a little forced to fit the view that the authors seem to have. I would suggest that they could be changed to flow together better.

Section V seems to again be a bit strange to read, the text is difficult to follow at times and the flow seems to diverge frequently. I don't think that the theories need to be described in detail here, provided they are referenced correctly it's enough to mention them and how they are relative.

The discussion seems to lack any relation to the data gathered, claims are made but are unsubstantiated most of the time. In saying this it reads easier than the previous sections but it could do with some more work. 

The conclusion doesn't really answer the questions presented in the introduction. As mentioned before it doesn't really follow the abstract either. The headers are a little distracting from the text and the flow is hard to follow. The one claim made in the conclusion isn't done so until the end of the whole section. 


\subsection*{Layout}
\subsection*{Grammar and readability}

% that's all folks
\end{document}


