\documentclass[conference]{IEEEtran}
\usepackage{cite}

\begin{document}
%
% paper title
% can use linebreaks \\ within to get better formatting as desired
\title{Outlook: What Would I Do Next?}

\author{\IEEEauthorblockN{
  John Burchell\IEEEauthorrefmark{1}}

  \IEEEauthorblockA{\IEEEauthorrefmark{1}john.a.burchell@gmail.com}
}

% make the title area
\maketitle

\begin{abstract}
This will briefly describe what I would do next e.t.c.

\cite{kitchenham2002preliminary}

\end{abstract}

\IEEEpeerreviewmaketitle

\section{Potential Improvements}

If there were to be a second iteration of the study, I would want to change the scope and some aspects of the methodology. Starting with the scope, I 
would like to perform the study over a much larger set of departments. Doing this would allow us to collect more results over a broader area of the company.
Doing this would allow us to make some better informed generalisations about 
how the implemented changes have affected employees at Ericsson. The main reason in doing this would be to improve our largest limitation; focusing on one department within Ericsson.

There would be some drawbacks, we would have to interview many more people, send out more questionnaires all of which take time from both ourselves and those working at Ericsson. It was already difficult enough to get the time for the interviews and respondents for the surveys that we did acquire. 

The parts of the methodology I'd like to change would be to have more qualitative data coming from direct interviews. We were limited by the fact that we only had one interview and as such had no other qualitative data to compare it with. 

Ideally I would like to have more interviews with people within the organisation. Specifically different kinds of management. In our study, our participant was middle management so we were missing the views of the upper management and those below our interviewee. Had we the opportunity to ask others, there might have been some more interesting results and therefore conclusions that we could contrasted and compared with the quantitative data.

All of these suggested improvements would require some changes to the methodology which was employed in the first iteration of the study.

\section{Suggested Methodology}

I would keep the outline of our study the same, that is the inclusion of a
literature review but we should also attempt to find some examples of change at large organisations. In my opinion we should focus on finding examples where types of change have come in different directions, i.e. top down or bottom up. This was an interesting point that came up during the interview we had and I think it could be interesting to explore it further.

With the suggestion of a potentially broader employees to collect data from a more formal method of selection for participation would be required. Kitchenham sums this up quite succinctly with the following ``The most convincing way of obtaining subjects is by random sampling''\cite{kitchenham2003principles}.

While I would object to calling them ``subjects'', I agree and would suggest that we employ some form of random sampling, most notably simple sampling or Systematic sampling. This is a common and well established method for finding generalities and trends for studies \cite{kitchenham2003principles}.

Next, I would suggest some changes to the questionnaires. As we were more interested in the beliefs held by the interviewees we should word our questions in a way to highlight these beliefs. Most importantly the should focus on how the changes that have been implemented have affected them specifically. A small note to make here is that we found that most of the respondents answered the first \%50 of the questions but a fair few did not finish the whole questionnaire, this is perhaps a indication that there were too many questions or that it was too long. This was also apparent in the answers themselves, for a good portion of the MDSD team marked a majority of the testing questions as not applicable to them. While this is probably unavoidable, especially for large organisations, we should strive to avoid this happening in a second iteration.

As mentioned previously, I would keep the interviews the same so that we can collect qualitative data but we would need to have multiple interviews this time. The questions for these interviews should be created so that we can probe differences in the opinions that may or may not be present in the quantitative data. Ideally we would have one person from upper, middle and lower management from each department.

\section{Relation to DIT035}

**Could ultimately ask the question, what types of change are more successful for an organisation? i.e. top down, bottom up e.t.c. We could contrast what people say against the metrics that Ericsson are actually using to see if they are inline - This is akin to the “frames” mentioned by Klein and Sorra, Congruence etc \cite{klein1996challenge}. There could be a mismatch between what the management perceive as successful as opposed to what the staff think is successful

Innovation values fit - Klein and Sorra

**Suggest following kotters principles, maybe find a few to suggest specifically \cite{kotter1995leading}

\section{Conclusion}

I would like to compare what the people think about the productivity of the new changes versus what Ericsson' metrics say. It might be interesting to see if they overlap but it would be more interesting to see that they didn't. This would probably require another study to investigate if it were found that there is a divergence but it would be interesting none-the-less.

\bibliography{outlook}{}
\bibliographystyle{IEEEtran}

% that's all folks
\end{document}


