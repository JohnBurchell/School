\documentclass[conference]{IEEEtran}
\usepackage{cite}

\begin{document}
%
% paper title
% can use linebreaks \\ within to get better formatting as desired
\title{Outlook: What Would I Do Next?}

\author{\IEEEauthorblockN{
  John Burchell\IEEEauthorrefmark{1}}

  \IEEEauthorblockA{\IEEEauthorrefmark{1}john.a.burchell@gmail.com}
}

% make the title area
\maketitle

\begin{abstract}
A Broader scope and improved methodology would allow further insights to be gained about how change is perceived and its success at Ericcson. By broadening the scope and including more departments in the study could allow  generalisations to be formed with appropriate evidence. Improving the methodology with the use of a formal sampling method such as random sampling or systematic sampling will aid the discovery of these potential generalisations. At the same time, more qualitative interviews will also help to reinforce the quantitative results but could also help to indicate discrepancies. These improvements and thus the new study can be discussed with regard to the content of DIT035. The results of this second iteration could be analysed with specific theories in mind and subsequent improvements to Ericsson could be suggested with specific change management strategies.
\end{abstract}

\IEEEpeerreviewmaketitle

\section{Introduction}

This paper will briefly explain my own opinions of what could be improved in a second iteration of our study as part of the DIT037 course. Starting with the potential improvements that could be made to the study, following with suggestions for improvements to the methodology and finally how it can all be related to the DIT035 course.

\section{Potential Improvements}

If there were to be a second iteration of the study, I would want to change the scope and some aspects of the methodology. Starting with the scope, I 
would like to perform the study over a much larger set of departments. Doing this would allow us to collect more results over a broader area of the company.
Doing this would allow us to make some better informed generalisations about 
how the implemented changes have affected employees at Ericsson. The main reason in doing this would be to improve our largest limitation; focusing on one department within Ericsson.

There would be some drawbacks, we would have to interview many more people, send out more questionnaires all of which take time from both ourselves and those working at Ericsson. It was already difficult enough to get the time for the interviews and respondents for the surveys that we did acquire. 

The parts of the methodology I'd like to change would be to have more qualitative data coming from direct interviews. We were limited by the fact that we only had one interview and as such had no other qualitative data to compare it with. 

Ideally I would like to have more interviews with people within the organisation. Specifically different kinds of management. In our study, our participant was middle management so we were missing the views of the upper management and those below our interviewee. Had we the opportunity to ask others, there might have been some more interesting results and therefore conclusions that we could contrasted and compared with the quantitative data.

All of these suggested improvements would require some changes to the methodology which was employed in the first iteration of the study.

\section{Suggested Methodology}

I would keep the outline of our study the same, that is the inclusion of a
literature review but we should also attempt to find some examples of change at large organisations. In my opinion we should focus on finding examples where types of change have come in different directions, i.e. top down or bottom up. This was an interesting point that came up during the interview we had and I think it could be interesting to explore it further.

With the suggestion of a potentially broader employees to collect data from a more formal method of selection for participation would be required. Kitchenham et. al sum this up quite succinctly with the following ``The most convincing way of obtaining subjects is by random sampling''\cite{kitchenham2002preliminary}. 

While I would object to calling them ``subjects'', I agree and would suggest that we employ some form of random sampling, most notably simple sampling or Systematic sampling. This is a common and well established method for finding generalities and trends for studies \cite{kitchenham2002preliminary}.

This would also mean that we would have to be in control of the distribution of the questionnaires. This was one of the major limitations of the initial study. We did not have control the distribution of the questionnaires and as such have little information pertaining to who received them and the amount that answered versus those that did not.

Next, I would suggest some changes to the questionnaires. As we were more interested in the beliefs held by the interviewees we should word our questions in a way to highlight these beliefs. Most importantly the should focus on how the changes that have been implemented have affected them specifically. A small note to make here is that we found that most of the respondents answered the first \%50 of the questions but a fair few did not finish the whole questionnaire, this is perhaps a indication that there were too many questions or that it was too long. This was also apparent in the answers themselves, for a good portion of the MDSD team marked a majority of the testing questions as not applicable to them. While this is probably unavoidable, especially for large organisations, we should strive to avoid this happening in a second iteration.

As we mentioned in the limitations of our study, our ordinal scale could be causing us to lose information from our questionnaire results when we convert the results into their numerical values. This is also something that Kitchenham et. al \cite{kitchenham2003principles} refer to in their Principles of Survey Research paper. They suggest that doing this conversion can cause the results to become misleading. To avoid this they give suggestions of other kinds of analysis that could be performed on an set of ordinal data. If we deem that the conversion is going to alter our results, then 

As mentioned previously, I would keep the interviews the same so that we can collect qualitative data but we would need to have multiple interviews this time. The questions for these interviews should be created so that we can probe differences in the opinions that may or may not be present in the quantitative data. Ideally we would have one person from upper, middle and lower management from each department.

\section{Relation to DIT035}

In order to relate this second iteration to the DIT035 course, the main research question could be turned into like the following: ``what types of change are more successful for a large organisation?''.

This could be explored by comparing the types of change that occur at Ericsson. Our study found that there are both top down and bottom up change occuring, so we could relate the bottom up changes to emergent changes as discussed by Orlikowski and Hofman \cite{orlikowski1997imporvisational}. This could likewise be contrasted against the top down or anticipated strategies also discussed by Orlikowski and Hofman \cite{orlikowski1997imporvisational}. The idea of the status quo, as suggested by Weinberg et. al \cite{weinberg1997quality} could be used to identify if these new ideas are being accepted and how the changes are actually occurring.

The idea of innovation values fit, as suggested by Klein and Sorra could also be explored by the results of the study \cite{klein1996challenge}. Was there a an environment for change? Do the values of the employees match those of the managers? If not, did the change fail as a result of this?

This idea could be further explored by investigating what Orlikowski and Gash call ``Frames'', the mindsets that the employees and employers have \cite{orlikowski1992changing}. Are the mindsets congruent or in-congruent? If there is a mismatch, why could this be?

\section{Conclusion}

Ultimately the second iteration would be an improvement over the second, we would much more information to draw even deeper generalisations about how change is managed at Ericsson. The analysis of the data would hopefully show some interesting relationships between the types of change and how their success is perceived by those it affects and by those that implement it.

I think it would be interesting to compare what the employees think about the productivity of the new changes versus what Ericsson's metrics say. It might be interesting to see if they overlap but it would be more interesting if they did not. In such an event another study would most likely be required to ascertain why the metrics say one thing and the employees another.

\bibliography{outlook}{}
\bibliographystyle{IEEEtran}

% that's all folks
\end{document}


